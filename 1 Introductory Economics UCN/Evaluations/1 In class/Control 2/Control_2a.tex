%% AMS-LaTeX Created by Wolfram Mathematica 7.0 : www.wolfram.com
\documentclass[letterpaper,12pt,final,titlepage]{article}
%PREAMBLE:
\usepackage[total={18cm,21cm},top=2cm, left=2cm]{geometry} %simplifica la definición de margenes
\usepackage[leqno]{amsmath}
\usepackage{latexsym}
\usepackage{amsmath, amssymb, graphics}
\usepackage{color} %agregar color a las partes del documento que se prefieran%
\usepackage[spanish,activeacute]{babel}
\usepackage[latin1]{inputenc}
\usepackage{setspace}
\usepackage[pdftex]{graphicx}
\usepackage{epstopdf}
\usepackage{graphicx} %sirve para insertar graficos en varios formatos%
\usepackage{epsfig,tocbibind}%para incluir figuras en .EPS de Stata
\usepackage{fancyhdr} %agrega topes y pies de pagina%
\usepackage{rotating} %por si quiero colocar un hoja horizontal
\usepackage{anysize} %sirve para definir margenes%
\usepackage{hyperref} %para crear enlaces dentro del propio documento o para insertar urls.
\usepackage[sort]{natbib} % Reference List
\usepackage{tabulary}
\usepackage{tabularx}
\usepackage{afterpage}
\newcommand{\mathsym}[1]{{}}
\newcommand{\unicode}{{}}

%%%%%%%%%%%%%%%%%%%%%%%%%%%%%%%%%%%%%%%%%%%%%%%%%%%%%%%%%%%%%%%%%%%%%%%%%%%%%%%%%%

\begin{document}
\begin{center}
\begin{large}
\textbf{Control econometría (EC402) número \#2:}
\end{large}\\
\textbf{Prof. Andrés Mauricio Castaño Zuluaga}
\end{center}
\textbf{El tiempo máximo para responder es de 20 min, no se permite sacar material, se debe responder en el espacio asignado.}\\
\\
\\
\textbf{1. Cuál es la diferencia entre el concepto de regresión utilizado por Galton y la definición moderna de regresión, de un ejemplo}
\\
\/
\textbf{1. Explique la(s) diferencias entre una relación estadística y una determinística}
\\
\\
\textbf{1. Explique la diferecia entre regresión y causalidad}
\\
\\
\textbf{1. Explique la diferencia entre regresión y correlación}

\end{document} 