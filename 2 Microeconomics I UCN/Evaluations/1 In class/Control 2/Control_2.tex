%% AMS-LaTeX Created by Wolfram Mathematica 7.0 : www.wolfram.com
\documentclass[letterpaper,12pt,final,titlepage]{article}
%PREAMBLE:
\usepackage[total={18cm,21cm},top=2cm, left=2cm]{geometry} %simplifica la definici�n de margenes
\usepackage[leqno]{amsmath}
\usepackage{latexsym}
\usepackage{amsmath, amssymb, graphics}
\usepackage{color} %agregar color a las partes del documento que se prefieran%
\usepackage[spanish,activeacute]{babel}
\usepackage[latin1]{inputenc}
\usepackage{setspace}
\usepackage[pdftex]{graphicx}
\usepackage{epstopdf}
\usepackage{graphicx} %sirve para insertar graficos en varios formatos%
\usepackage{epsfig,tocbibind}%para incluir figuras en .EPS de Stata
\usepackage{enumerate}
\usepackage{fancyhdr} %agrega topes y pies de pagina%
\usepackage{rotating} %por si quiero colocar un hoja horizontal
\usepackage{anysize} %sirve para definir margenes%
\usepackage{hyperref} %para crear enlaces dentro del propio documento o para insertar urls.
\usepackage[sort]{natbib} % Reference List
\usepackage{tabulary}
\usepackage{tabularx}
\usepackage{afterpage}
\newcommand{\mathsym}[1]{{}}
\newcommand{\unicode}{{}}

%%%%%%%%%%%%%%%%%%%%%%%%%%%%%%%%%%%%%%%%%%%%%%%%%%%%%%%%%%%%%%%%%%%%%%%%%%%%%%%%%%

\begin{document}
\begin{center}
\begin{large}
\textbf{Control econometr�a (EC402) n�mero \#2:}
\end{large}\\
\textbf{Prof. Andr�s Mauricio Casta�o Zuluaga}
\end{center}
\textbf{El tiempo m�ximo para responder es de 20 min, no se permite sacar material, se debe responder en el espacio asignado.}\\
\\
\\
\textbf{1. Cu�l es la diferencia entre el concepto de regresi�n utilizado por Galton y la definici�n moderna de regresi�n, de un ejemplo}\\
\\
\\
\textbf{2. Explique la(s) diferencias entre una relaci�n estad�stica y una determin�stica}\\
\\
\\
\textbf{3. Explique la(s) diferencia entre regresi�n y causalidad, de un ejemplo}\\
\\
\\
\textbf{4. Explique la(s) diferencia entre regresi�n y correlaci�n, de un ejemplo}\\
\\
\\
\textbf{5. Clasifique y de un ejemplo de las escalas de medici�n de las variables}

\end{document}
