\documentclass[letterpaper,12pt,final,titlepage]{article}
%PREAMBLE:
\usepackage[total={18cm,21cm},top=2cm, left=2cm]{geometry} %simplifica la definici�n de margenes
\usepackage[leqno]{amsmath}
\usepackage{latexsym}
\usepackage{amsmath, amssymb, graphics}
\usepackage{color} %agregar color a las partes del documento que se prefieran%
\usepackage[spanish,activeacute]{babel}
\usepackage[latin1]{inputenc}
\usepackage{setspace}
\usepackage[pdftex]{graphicx}
\usepackage{epstopdf}
\usepackage{graphicx} %sirve para insertar graficos en varios formatos%
\usepackage{epsfig,tocbibind}%para incluir figuras en .EPS de Stata
\usepackage{fancyhdr} %agrega topes y pies de pagina%
\usepackage{rotating} %por si quiero colocar un hoja horizontal
\usepackage{anysize} %sirve para definir margenes%
\usepackage{hyperref} %para crear enlaces dentro del propio documento o para insertar urls.
\usepackage[sort]{natbib} % Reference List
\usepackage{tabulary}
\usepackage{tabularx}
\usepackage{afterpage}
\usepackage{afterpage}
\usepackage{atbegshi}% http://ctan.org/pkg/atbegshi
\newcommand{\mathsym}[1]{{}}
\newcommand{\unicode}{{}}
\providecommand{\abs}[1]{\lvert#1\rvert}
\providecommand{\norm}[1]{\lVert#1\rVert}
\AtBeginDocument{\AtBeginShipoutNext{\AtBeginShipoutDiscard}}




%%%%%%%%%%%%%%%%%%%%%%%%%%%%%%%%%%%%%%%%%%%%%%%%%%%%%%%%%%%%%%%%%%%%%%%%%%%%%%%%%%

\begin{document}
\title{\textbf{Ayudant�a \# 1 Microeconom�a I \\
		(EC-301 Y EC-210)\\
}}
\begin{center}
\author{Prof. Andr�s M. Casta�o Zuluaga\\
\\
Ayudantes:\\
\\
Stacy Amas Morales\\
Jose Federsffield Ugalde\\
\\
Microeconom�a I (EC301 y EC-210)\\
Universidad Cat\'olica del Norte\\}


\end{center}

\date{\today}
\maketitle

\section{Repaso elementos de oferta y demanda}

\subsection{Equilibrio de mercado}

\begin{enumerate}
\item Suponga que queremos analizar el mercado de los cacahuetes y que, partiendo del an�lisis estad�stico de datos hist�ricos, concluimos que la calidad de cacahuetes demandada cada semana (Q, medida en fanegas) depende del precio de los cacahuetes (P, medido en d�lares por fanega) siguiendo la ecuaci�n
$$Q_{D}=1000-100P$$
Ahora suponga que la cantidad ofertada de cacahuetes tambi�n depende del precio:
$$Q_{S}=-125+125P$$
De acuerdo a lo anterior determine el equilibrio de mercado
\bigskip
\item Suponga ahora que la demanda de cacahuetes aumenta hasta:
$$Q_{D}^{'}=1450-100P$$
determine el nuevo equilibrio de mercado, y grafique los dos escenarios.
\end{enumerate}


\subsection{Elasticidad}

\begin{enumerate}
	\item Considere el mercado de cigarrillos donde la demanda est� compuesta por dos grupos. Los del Grupo A son los ''adictos'' al 
	cigarrillo y los del Grupo B son los que fuman ''de vez en cuando''. El A es un grupo peque�o frente al B. Dibuje la curva de demanda de cada grupo, explicando su diferencia en ubicaci�n y elasticidad.
	\bigskip
	\item Suponga una curva de oferta normal que atiende el total del mercado. Marque en el gr�fico el equilibrio del mercado.
	\bigskip
	\item Suponga la siguiente funci�n de demanda:
	
	$$Q_{x} = 200 - 2P_{x} - 3P_{y}$$
	
	donde $Q_{x}$ es la cantidad demandada de x, $P_{x}$ es el precio de x y $P_{y}$
	es el precio de y. Suponga que s�lo se conocen cuatro puntos de la funci�n de demanda:
		\begin{center}
			\begin{tabular}{|c|c|c|}
				\hline
				% after \\: \hline or \cline{col1-col2} \cline{col3-col4} ...
				Punto & $P_{x}$ & $P_{y}$ \\
				\hline
				A & 6 & 5 \\
				B & 8 & 7 \\
				C & 10 & 10 \\
				D & 8 & 5 \\
				\hline
			\end{tabular}
		\end{center}
Calcule donde sea posible la $e_{PQ}^{QD}$, utilizando la formula de elasticidad arco. Luego utilizando los datos del punto anterior calcule la $E_{PR}^{QD}$, donde sea posible. Interprete todos los resultados obtenidos en ambos puntos, e indique a que tipo de bienes hace referencia cada ejemplo.
\end{enumerate}


\section{Introducci�n a la Teor�a del Consumidor}

\subsection{Restricci�n presupuestaria}

\begin{enumerate}
	\item La restricci�n presupuestaria de Pedro viene determinada por $m=500$; $p_{1}=1$; $p_{2}=2$. Explique como cambia el conjunto presupuestario si el gobierno aplica un impuesto espec�fico de 0.1 al bien 1.
	\bigskip
	\item Si el gobierno aplica un impuesto ad-valorem de 20\% al bien 1.
	\bigskip
	\item Si el gobierno aplica un impuesto de suma alzada de 100.
	\bigskip
	\item Si la pendiente de la restricci�n presupuestaria (precio relativo) es 1.
\end{enumerate}
\end{document}

