%% AMS-LaTeX Created by Wolfram Mathematica 7.0 : www.wolfram.com
\documentclass[letterpaper,12pt,final,titlepage]{article}
%PREAMBLE:
\usepackage[total={18cm,21cm},top=2cm, left=2cm]{geometry} %simplifica la definici�n de margenes
\usepackage[leqno]{amsmath}
\usepackage{latexsym}
\usepackage{amsmath, amssymb, graphics}
\usepackage{color} %agregar color a las partes del documento que se prefieran%
\usepackage[spanish,activeacute]{babel}
\usepackage[latin1]{inputenc}
\usepackage{setspace}
\usepackage[pdftex]{graphicx}
\usepackage{epstopdf}
\usepackage{graphicx} %sirve para insertar graficos en varios formatos%
\usepackage{epsfig,tocbibind}%para incluir figuras en .EPS de Stata
\usepackage{enumerate}
\usepackage{fancyhdr} %agrega topes y pies de pagina%
\usepackage{rotating} %por si quiero colocar un hoja horizontal
\usepackage{anysize} %sirve para definir margenes%
\usepackage{hyperref} %para crear enlaces dentro del propio documento o para insertar urls.
\usepackage[sort]{natbib} % Reference List
\usepackage{tabulary}
\usepackage{tabularx}
\usepackage{afterpage}
\newcommand{\mathsym}[1]{{}}
\newcommand{\unicode}{{}}

%%%%%%%%%%%%%%%%%%%%%%%%%%%%%%%%%%%%%%%%%%%%%%%%%%%%%%%%%%%%%%%%%%%%%%%%%%%%%%%%%%

\begin{document}
\begin{center}
\begin{large}
\textbf{Control econometr�a (EC402) n�mero \#1:}
\end{large}\\
\textbf{Prof. Andr�s Mauricio Casta�o Zuluaga}
\end{center}
\textbf{El tiempo m�ximo para responder es de 30 min, no se permite sacar material, se debe responder en el espacio asignado.}\\

\textbf{1. Explique la diferencia entre una variable aleatoria discreta y una variable aleatoria continua, de un ejeplo de cada una (menos de 6 l�neas).}
\bigskip

\textbf{2. Explique en qu� consiste una funci�n de densidad de probabilidad condicional, haga un gr�fico y de un ejemplo.}
\bigskip

\textbf{3. Explique porqu� una funci�n de distribuci�n de probabilidad puede definirse a partir de sus momentos. Defina tres momentos de una distribuci�n.}
\bigskip

\textbf{4. Considere la siguiente funci�n de densidad de probabilidad discreta (FDPD):}\\
\\
\begin{tabular}{|c|c|c|c|}
  \hline
  % after \\: \hlineskip or \cline{col1-col2} \cline{col3-col4} ...
  x & -2 & 1 & 2 \\
  f(x) & $\frac{5}{8}$ & $\frac{1}{8}$ & $\frac{2}{8}$ \\
  \hline
\end{tabular}
\\
\\
\\
A. Calcule el valor esperado ($E(X)$) y la varianza ($Var(X)$). Pista1: $E(x)=E(X)=\sum_{x}xf(x)$, pista2: $var(X)=E(X^{2})-(E(X))^{2}$.
\\
\\
\textbf{5. Considere la siguiente funci�n de densidad de probabilidad continua (FDPC):}\\
\begin{equation}
f(x)=\frac{x^{2}}{9}
\end{equation}
\\
donde : $(0\leq x\leq 3)$
\\
\\
A. Calcule el valor esperado y la varianza de dicha funci�n de probabilidad. Pista1: $E(x)=\int^{b}_{a}xf(x)dx$, pista2: $var(X)=\int^{b}_{a}(x-\mu)^{2}f(x)dx$, pista3 $var(X)=E(X^{2})-(E(X))^{2}$, pista4 $\int x^{a}dx=\frac{1}{a+1}x^{a+1}+C$

\end{document} 