\documentclass[letterpaper,12pt,final,titlepage]{article}
%PREAMBLE:
\usepackage[total={18cm,21cm},top=2cm, left=2cm]{geometry} %simplifica la definici�n de margenes
\usepackage[leqno]{amsmath}
\usepackage{latexsym}
\usepackage{amsmath, amssymb, graphics}
\usepackage{color} %agregar color a las partes del documento que se prefieran%
\usepackage[spanish,activeacute]{babel}
\usepackage[latin1]{inputenc}
\usepackage{setspace}
\usepackage[pdftex]{graphicx}
\usepackage{epstopdf}
\usepackage{graphicx} %sirve para insertar graficos en varios formatos%
\usepackage{epsfig,tocbibind}%para incluir figuras en .EPS de Stata
\usepackage{fancyhdr} %agrega topes y pies de pagina%
\usepackage{rotating} %por si quiero colocar un hoja horizontal
\usepackage{anysize} %sirve para definir margenes%
\usepackage{hyperref} %para crear enlaces dentro del propio documento o para insertar urls.
\usepackage[sort]{natbib} % Reference List
\usepackage{tabulary}
\usepackage{tabularx}
\usepackage{afterpage}
\newcommand{\mathsym}[1]{{}}
\newcommand{\unicode}{{}}

%%%%%%%%%%%%%%%%%%%%%%%%%%%%%%%%%%%%%%%%%%%%%%%%%%%%%%%%%%%%%%%%%%%%%%%%%%%%%%%%%%

\begin{document}
\title{\textbf{Ejercicios Econometr�a\\
Ayudant�a \#4}}
\begin{center}
\author{Prof. Andr\'es Casta�o Zuluaga\\
\\
Ayudantes:\\
\\
Josefa Pellejero Maranguni\v{c}\\
Mariana Camila Nadal Fernandez\\
\\
Econometr�a I (EC402)\\
Ingenier�a Comercial\\
Universidad Cat\'olica del Norte\\}

\end{center}

\date{\today}
\maketitle

\section{Demostraciones �tiles de las propiedades num�ricas del ajuste MCO}

\subsection{Demuestre que $\bar{\mu}=0$}
\noindent Para comenzar a demostrar que el valor medio de los residuos es igual a cero, partimos de la definici�n conocida:
$$\mu_{i}=Y_{i}-\hat{Y}_{i}$$
Dado que por la definici�n de la ecuaci�n de regresi�n muestral estoc�stica se define $\hat{Y}_{i}=\hat{\beta}_{1}+\hat{\beta}_{2}X_{i}$, entonces:
$$\mu_{i}=Y_{i}-\hat{\beta}_{1}-\hat{\beta}_{2}X_{i}$$
Aplicando sumatoria a ambos lados obtenemos:
$$\sum_{i=1}^{n}\mu_{i}=\sum_{i=1}^{n}Y_{i}-n\hat{\beta}_{1}-\hat{\beta}_{2}\sum_{i=1}^{n}X_{i}$$
Si dividimos ambos lados entre n se obtiene:
$$\bar{\mu}=\bar{Y}-\hat{\beta}_{1}-\hat{\beta}_{2}\bar{X}$$
Finalmente si reemplazamos la definici�n del intercepto se obtiene:
$$\bar{\mu}=\bar{Y}-\bar{Y}+\hat{\beta}_{2}\bar{X}-\hat{\beta}_{2}\bar{X}=0$$

\subsection{Demuestre que $\bar{\hat{Y}}=\bar{Y}$}
$$\mu_{i}=Y_{i}-\hat{Y}_{i}$$
Aplicando sumatoria a ambos lados:
$$\sum_{i=1}^{n}\mu_{i}=\sum_{i=1}^{n}Y_{i}-\sum_{i=1}^{n}\hat{Y}_{i}$$
Dividiendo entre n:
$$\bar{\mu}=\bar{Y}-\bar{\hat{Y}}$$
Por lo cual dado que $\bar{\hat{\mu}}=0$, entonces:
$$\bar{Y}=\bar{\hat{Y}}$$

\subsection{Demuestre que $\sum_{i=1}^{n}\hat{Y_{i}}\mu_{i}=0$}
$$\sum_{i=1}^{n}\hat{Y_{i}}\mu_{i}=\sum_{i=1}^{n}(\hat{\beta}_{1}+\hat{\beta}_{2}X_{i})\mu_{i}$$
Resolviendo se obtiene:
$$\sum_{i=1}^{n}\hat{Y_{i}}\mu_{i}=\hat{\beta}_{1}n\bar{\mu}+\hat{\beta}_{2}\sum_{i=1}^{n}X_{i}\mu_{i}$$
Ahora dado que $\frac{\sum_{i=1}^{n}\mu_{i}}{n}=\bar{\mu}$, reemplazamos en la ecuaci�n este resultado y obtenemos:
$$\sum_{i=1}^{n}\hat{Y_{i}}\mu_{i}=\hat{\beta}_{1}\sum_{i=1}^{n}\mu_{i}+\hat{\beta}_{2}\sum_{i=1}^{n}X_{i}\mu_{i}$$
Dado que $\sum_{i=1}^{n}\mu_{i}=0$ y  $\sum_{i=1}^{n}X_{i}\mu_{i}=0$, entonces:
$$\sum_{i=1}^{n}\hat{Y_{i}}\mu_{i}=0$$


\section{Demostraciones Bondad de Ajuste}

\subsection{Demuestre que $R^{2}=1-\frac{SRC}{STC}$}
\noindent La variabilidad total de la $Y_{i}$ observadas se puede expresar a trav�s de la suma total de cuadrados (STC) o suma cuadr�tica total (SCT):
$$SCT=\sum^{n}_{i=1}(Y_{i}-\bar{Y})^{2}$$
Descomponemos la variabilidad en:
$$STC=\sum^{n}_{i=1}(\hat{Y}_{i}+\mu_{i}-\bar{Y})^{2}=\sum^{n}_{i=1}((\hat{Y}_{i}-\bar{Y})+\mu_{i})^{2}$$
$$=\sum^{n}_{i=1}(\hat{Y}_{i}-\bar{Y})+\sum^{n}_{i=1}\mu_{i}^{2}+2\sum^{n}_{i=1}(\hat{Y}_{i}-\bar{Y})\mu_{i}$$
$$=\sum^{n}_{i=1}(\hat{Y}_{i}-\bar{Y})^{2}+\sum^{n}_{i=1}\mu_{i}^{2}+2\sum^{n}_{i=1}\mu_{i}\hat{Y}_{i}-2\bar{Y}\sum^{n}_{i=1}\mu_{i}$$
Si se satisface el supuesto de que las $X_{i}$ no son todas iguales, $STC>0$ y por lo tanto.
$$\frac{STC}{STC}=\frac{SEC}{STC}+\frac{SRC}{STC}$$
$$\frac{SEC}{STC}+\frac{SRC}{STC}=1$$
$\frac{SEC}{STC}$ es la proporci�n de la variabilidad total explicada por el modelo $= R^{2}$ o coeficiente de determinaci�n:
$$R^{2}=\frac{SEC}{STC}$$
� bien:
$$R^{2}=1-\frac{SRC}{STC}$$
$\frac{SRC}{STC}$ es la proporci�n de la variabilidad total NO explicada por el modelo.

\subsection{demuestre que la ra�z cuadrada de $R^{2}$ es igual a $r_{Y, \hat{Y}}$}
Se parte de la definici�n del coeficiente de correlaci�n entre dos variables:
$$r_{Y,\hat{Y}}=\frac{\sum_{i}^{n}(Y_{i}-\bar{Y})(\hat{Y_{i}}-\bar{Y})}{\sqrt{\sum_{i}^{n}(Y_{i}-\bar{Y})^{2}\sum_{i}^{n}(\hat{Y_{i}}-\bar{Y})^2}}$$

\end{document}
